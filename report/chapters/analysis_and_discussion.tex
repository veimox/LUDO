%!TEX root = ../report.tex
\section{Analysis and Discussion} % (fold)
\label{sec:analysis_and_discussion}
The results shows that the ANNGA player wins in all the tournaments against the AIs already implemented in the code and also against the AIs from other students. 
This have been achieved with 500 generations what implies around 3 minutes of processing time.
This factor is important.
Also, the player implemented has not a \"learning limit\", meaning that the goal is to win not learning from another players.
It tries to find the optimal solution for the specific players.
This is congruent with a luck based game because there is no an optimal solution. \\

Despite probably the Q-learning with GA was potentially a better player due to it stores more information about the game, the ANNGA has shown to be better and in less time.
This could be because the Q-learner has not being trained enough and here is when the time factor becomes important.
It doesn't matter all the possibilities that are considered if there is no time to test them.
Probably, the best strategy is not try to study all the possibilities but only the best ones.\\

Now the ANNGA player has a fixed chromosome size, but a further work would be make this chromosome plastic.
Depending on the game, maybe the situation in which the brick can hit an opponent, be in a star and be close from another enemy only appears the 5\% of the times.
When applying an specific gene to that situation maybe resources are being wasted.
An improvement to the player would be to make it modify its chromosome online based on what it consider relevant to evolve.
The main advantage of an ANN is that given an input it will always give an output, so despite the net has not been trained for that situation it can give a solution.\\

Other improvements can be made in the GA.
The population now is really small (8), so increasing the size would increase the performance.
This limitation is due to that now the CPU threads are used but perhaps the program could be implemented using GPU cores instead, where millions of parallel games can be played ad the same time.

% section analysis_and_discussion (end)